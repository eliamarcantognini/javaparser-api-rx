\section{Analisi del problema}

Si vuole implementare una libreria che metta a disposizione una serie di funzionalità minimali per l'analisi di un progetto sviluppato utilizzando il linguaggio Java.
%
Le funzionalità fornite dalla libreria devono dar la possibilità di:

\begin{enumerate}
    \item Analizzare un'interfaccia, restituendo il nome stesso e un elenco dei nomi dei metodi contenuti al suo interno.
    
    \item Analizzare una classe, restituendo il suo nome e un elenco dei metodi in essa contenuti, comprensivi di alcune informazioni ritenute essenziali: nome, modificatori (public, private, ecc.) e la loro posizione all'interno del file, specificando linea di inizio e di fine.
    
    \item Analizzare un package, riportando un analisi delle interfacce e delle classi al suo interno.
    
    \item Analizzare il progetto stesso, riportando un'analisi dei suoi package.
\end{enumerate}

Queste analisi si possono considerare a cascata, dato che le ultime si compongono di quelle precedenti.

Ulteriore requisito è quello di realizzare la libreria in modo asincrono, utilizzando un event-loop come architettura di riferimento.
%
In questo modo sarà possibile utilizzare la libreria in modo non bloccante, grazie al meccanismo delle \textit{Future}.

Inoltre, deve essere possibile poter interrompere in qualsiasi momento l'esecuzione di un'analisi precedentemente richiesta.

Infine, si intende realizzare una semplice applicazione provvista di GUI per poter testare le funzionalità messe a disposizione dalla libreria mostrando a schermo i risultati delle varie analisi.

\section{Strategia risolutiva}

La soluzione implementata, come specificato nei requisiti, si basa su programmazione asincrona in cui le varie operazioni vengono eseguite tramite l'utilizzo di un event-loop.
%
Per realizzare questo comportamento si è utilizzato il \textit{framework} \textit{Vertx}, che permette appunto di realizzare funzionalità asincrone in modo semplice, senza doversi preoccupare di molti aspetti di basso livello, come l'utilizzo effettivo dei \textit{Thread}.

Per quanto riguarda l'analisi degli elementi di un progetto Java si è utilizzato la libreria \textit{Java Parser}, che permette di creare degli \textit{Abstract Sintaxt Tree} (AST) a partire da un file .java e visitare i sui nodi per recuperare le informazioni di interesse.
%
Nello specifico, la visita viene effettuata tramite dei \textit{Visitor} in cui è possibile specificare il comportamento da adottare quando viene visitato un certo tipo di nodo.

Utilizzo di un \textit{event-loop} esterno che dovrà gestire i vari eventi che nel nostro caso sono le chiamate ai metodi per ottenere un determinato report.

La funzionalità analyzeProject lancerà una serie di eventi in base al tipo di file che si incontra durante la visita del progetto, ognuno dei quali relativo ad un certo report.

Si pensa ad una possibile soluzione in cui per ogni report si va a creare un event-loop personale in cui gestire i singoli elementi da analizzare. Tale soluzione diventa interessante se si riuscisse a fare in modo che l'event-loop generale avvia un event-loop figlio per ottenere un certo report e poi va a gestire un altro evento (la richiesta di un altro report). Deve essere l'event-loop figlio a comunicare di aver terminato al padre per fargli raccogliere il risultato. 

\section{Architettura ed implementazione}

I metodi getReport devono essere asincroni quindi tornano una Future del tipo relativo al report che si vuole ottenere (per esempio Future<ClassReport>)